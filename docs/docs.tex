\documentclass{article}
\usepackage{graphicx}

\begin{document}

\title{Devpi In The Cloud}
\author{Lee Hicks}

\maketitle

\section{What is this?}
This project is designing and implementing a Devpi server for pypi caching via Amazon Web Services (AWS).

\section{Versions}
There are multiple versions, all being a derivative of the base version.

\begin{itemize}
    \item Base - the bare minimum to get a devpi cache up in running via AWS 
    \item V1 - Base + Elastic Load Balancing (ELB) and AutoScaling
    \item V2 - still haven't decided yet
\end{itemize}

\section{Version - Base}
The base version is enough to deploy a devpi cache in AWS via CloudFormation (CF).

\subsection{Architecture}
For a simple devpi server all that is needed is devpi(duh). Devpi has a built in server via Bottle so with a few
commands devpi is installed and running. While the built in server is handy in my testing it seemed only to
serve requests that originate from localhost. The more permanent deployment is to put nginx in front of the
devpi cache.

TODO SIMPLE FIGURE OF DESIGN 

\subsection{Implementation}
To deploy a simple devpi server a CF script was written. The CF script
is pretty vanilla with starting a micro EC2 instance of Amazon Linux and applying the standard security policies of
port 22 and 80 open. CF Cloud-Init is used to install puppet. I also took advantage of the UserData field which
allows shell commands to be processed. In this case I use the userdata to fetch the initial puppet script and apply it.

TODO SIMPLE FIGURE OF BOOT PROCESS

\subsection{Puppet}
\subsubsection{How is it used?}
I debated quite a while on how I was going to use puppet for configuration. In the end I came to the conclusion that
running masterless puppet fit this application the best.  

The use case:
\begin{itemize}
    \item Security isn't really a tight concern since these are pypi caches. Unfortunately it is possible for an attacker
        to comprise the package at the cache. I would argue that the pypi cache itself is a larger target since most of
        the packages are signed and can't be verified, that I can tell.
    \item The EC2 instances would be brought up and down at the drop of the hat
    \item They are only serving < 100 servers, no public cache
    \item Design to be 100\% up time as possible 
    \item Didn't need any reports
    \item I don't want to touch it once deployed, unless I have too... 
    \item Most importantly KISS - KEEP IT SIMPLE STUPID
\end{itemize}

\subsubsection{Going Masterless}

Using a puppet master PM requires at least a dedicated ec2 to be on the safe side. This could be ran on an instance
that is also serving devpi. Only problem is when under load any extra processing seems to kill micro instance performance.
More importantly, a PM now becomes a single point of failure. If the PM goes down and instances
come online, either replacing dead instances or handling extra load, no PM to pass along an Inuit manifest is now a show stopper.
So to reduce that risk one could put a puppet master behind another ELB with dedicated instance, increasing cost, and also adds 
a whole host of new problems like master slave PMs, how do we enforce that, when the master goes down how does trust get 
transferred to slave PM, does it scale? A rabbit whole of issues to solve that masterless puppet steers around.

\subsubsection{Implementation}

\begin{enumerate}
    \item Puppet init stored on S3 bucked
    \item Instanced fired up via CF using CF-init and userdata script
    \item CF-init ensures puppet is installed
    \item Userdata script grabs the init .pp file and applies it
    \item Puppet applies the manifest and also creates a cron job to fetch and apply every X minutes (our case 5) 
\end{enumerate}

This allows any AMI that is RHEL based, aka yum, use this process without requiring us to bake our own AMIs. This can
be extended very easily to any AMI that CF-init supports.

Since the cron jobs runs every X minutes applying updates to manifest is a simple as uploading a new manifest to the S3 bucket
and sip a beer.

\subsubsection{Woes/Drawbacks}
As mentioned above one of the drawbacks is only vanilla puppet modules can be used, if you need modules your own your own.
Example of this is setting up the cron job. The tao of puppet seems to be that puppet defines the state the system should
be rather than using rules and executing shell commands. Unfortunately, at this moment I don't know how to tell puppet how
to install modules, and google didn't turn up anything. So I had to use shell commands in the UserData to add the cron module
before puppet apply.

To complicate matters, puppet installs modules in several places, /etc/puppet/modules:/usr/share/puppet/modules. When 
/etc/puppet/modules wasn't found it would halt right there without trying the any of the other paths defined. To this day I
don't know why that is. So to fix I, once again, added in userdata to create the darn path so puppet install module wouldn't
complain.

Another issue is security, currently at the time of writing the manifest is on a public S3 bucket. For this usecase I don't
believe this is an issue since no sensitive info in currently in the manifest. For more sensitive manifests I do plan on adding
S3 permissions so only users of AWS, and instances, can access the S3 bucket.

\section{Analysis}
\subsubsection{Performance}
\subsubection{Cost}
\subsubection{Conclusion}
The best use case for this is to have a puppet configurable devpi cache that is always on, and possible shared with a few devs. 

\section{Version - V1}

\subsection{Architecture}

\subsection{Implementation}

\subsection{Puppet}
\subsubsection{How is it used?}
I debated quite a while on how I was going to use puppet for configuration. In the end I came to the conclusion that
running masterless puppet fit this application the best.  

The use case:
\begin{itemize}
    \item The EC2 instances would be brought up and down at the drop of the hat
    \item They are only serving < 100 servers, no public cache
    \item Design to be 100\% up time as possible 
    \item I don't want to touch it once deployed, unless I have too... 
    \item Most importantly KISS - KEEP IT SIMPLE STUPID
\end{itemize}

\subsubsection{Implementation}
The use case:

\subsubsection{Woes/Drawbacks}

\section{Analysis}
\subsubsection{Performance}
\subsubection{Cost}
\subsubection{Conclusion}

\end{document}
